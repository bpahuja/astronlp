\begin{table}[htbp]
\centering
\caption{Cluster ID to Label Mapping}
\label{tab:cluster_labels}
\begin{tabular}{|c|l|}
\hline
\textbf{Cluster ID} & \textbf{Cluster Label} \\
\hline
0 & Noise Mitigation Techniques \\
\hline
1 & Nucleation and Bubble Detection Methods \\
\hline
2 & Dendrogram Analysis Methodology \\
\hline
3 & Hypergeometric Function Analysis \\
\hline
4 & Piezo-electric actuator systems \\
\hline
5 & Void Identification Techniques \\
\hline
6 & Spitzer Data Processing \\
\hline
7 & SPH simulations and methodologies \\
\hline
8 & Metropolis-Hastings MCMC \\
\hline
9 & Leapfrog Integration Methods \\
\hline
10 & Ornstein-Uhlenbeck Process Modeling \\
\hline
11 & Volcanic Activity Analysis \\
\hline
12 & Percolation and Cluster Analysis \\
\hline
13 & Fitness Function Optimization \\
\hline
14 & Sputtering and Grain Dynamics \\
\hline
15 & Genetic Algorithm Optimization \\
\hline
16 & Edgeworth expansion methods \\
\hline
17 & Particle Swarm Optimization (PSO) \\
\hline
18 & Vector Network Analyzer Measurements \\
\hline
19 & Conjugate Gradient Methodology \\
\hline
20 & Chromosome Map Analysis \\
\hline
21 & GRMHD numerical simulations \\
\hline
22 & Avalanche Charge Detection Methods \\
\hline
23 & Coupled Cluster Methods \\
\hline
24 & Neural Network Analysis \\
\hline
25 & SVD and Linear Inversion Techniques \\
\hline
26 & Cosmic Web Classification Methods \\
\hline
27 & Cholesky decomposition methods \\
\hline
28 & Raman Spectroscopy Techniques \\
\hline
29 & Density Functional Theory (DFT) Methods \\
\hline
30 & Genus Curve Analysis \\
\hline
31 & Signal Reconstruction Techniques \\
\hline
32 & Renormalization Group Methods \\
\hline
33 & Kalman Filter Methodology \\
\hline
34 & Mass Spectrometry Techniques \\
\hline
35 & Propagator Analysis Techniques \\
\hline
36 & Monte Carlo and Residual Permutation \\
\hline
37 & Spectroscopic Analysis of HgMn Stars \\
\hline
38 & Independent Component Analysis (ICA) \\
\hline
39 & Kelvin-Helmholtz instability simulations \\
\hline
40 & Subhalo mass estimation methods \\
\hline
41 & Beamforming and Signal Processing \\
\hline
42 & Tessellation and Quadrature Methods \\
\hline
43 & Random Walk Models \\
\hline
44 & DEM Inversion and Analysis \\
\hline
45 & Spectroastrometry Techniques \\
\hline
46 & Lidar Atmospheric Measurement \\
\hline
47 & MHD simulations of Galactic structures \\
\hline
48 & Shapelet Decomposition Methods \\
\hline
49 & GPU-based N-body simulations \\
\hline
50 & Photodiode Calibration Techniques \\
\hline
51 & Buoyancy and Stability Analysis \\
\hline
52 & Monochromator and Wavelength Selection \\
\hline
53 & Geopotential Error Analysis \\
\hline
54 & Lyapunov Exponent Analysis \\
\hline
55 & PV Diagram Analysis \\
\hline
56 & Smoothed Particle Hydrodynamics (SPH) \\
\hline
57 & Beamforming Techniques in Radio Astronomy \\
\hline
58 & Butterworth Spatial Filtering \\
\hline
59 & Higgs potential analysis \\
\hline
60 & Subarray Mode Observations \\
\hline
61 & Bombardment and Alignment Mechanisms \\
\hline
62 & X-ray source analysis methods \\
\hline
63 & CFL Condition Numerical Methods \\
\hline
64 & Orbital dynamics and imaging \\
\hline
65 & Autoregressive Time-Series Analysis \\
\hline
66 & Gyrosynchrotron Emission Analysis \\
\hline
67 & Voronoi Tessellation Methods \\
\hline
68 & Adaptive Mesh Refinement (AMR) Simulations \\
\hline
69 & Quantum Chemistry Calculations \\
\hline
70 & Central Limit Theorem Applications \\
\hline
71 & Parallelized Numerical Simulations \\
\hline
72 & Density-Based Clustering Methods \\
\hline
73 & Cryogenic Systems and Measurements \\
\hline
74 & Delayed Detonation Models \\
\hline
75 & Polarisation Analysis Instrumentation \\
\hline
76 & Plume Analysis and Imaging \\
\hline
77 & Random Number Generation Methods \\
\hline
78 & Brane perturbation analysis \\
\hline
79 & GALPROP model usage \\
\hline
80 & PHOEBE and Wilson-Devinney methods \\
\hline
81 & PCA and HEXTE data analysis \\
\hline
82 & Principal Component Analysis \\
\hline
83 & Sulphur Abundance Estimation \\
\hline
84 & Doppler Tomography Methods \\
\hline
85 & Statistical Isotropy Analysis \\
\hline
86 & Spectral Analysis of MgH Lines \\
\hline
87 & Radioheliograph Observations \\
\hline
88 & Ram Pressure Stripping Analysis \\
\hline
89 & Levenberg-Marquardt fitting \\
\hline
90 & Meridional Circulation Analysis \\
\hline
91 & Lax-Wendroff numerical methods \\
\hline
92 & Wavefunction Expansion Methods \\
\hline
93 & Compact Binary Coalescence Analysis \\
\hline
94 & Torque and Perturbation Analysis \\
\hline
95 & Gaussian Mixture Models \\
\hline
96 & Collision Simulation Analysis \\
\hline
97 & Crater Impact Modeling \\
\hline
98 & LNA Measurement and Calibration \\
\hline
99 & Deblending and Spectral Analysis \\
\hline
100 & SZE Observational Techniques \\
\hline
101 & Axion signal analysis \\
\hline
102 & Inflationary Cosmology Analysis \\
\hline
103 & Gauss-Hermite analysis \\
\hline
104 & Runge-Kutta numerical integration \\
\hline
105 & Gaussian Process Modeling \\
\hline
106 & Detrending and Transit Analysis \\
\hline
107 & Hamiltonian Dynamics Analysis \\
\hline
108 & Variational Equations Analysis \\
\hline
109 & General Relativity Approaches \\
\hline
110 & Multigrid Methodology \\
\hline
111 & Stereoscopic Analysis Frameworks \\
\hline
112 & Seismology and Noise Analysis \\
\hline
113 & Sunyaev-Zel'dovich Effect Analysis \\
\hline
114 & Upwind Schemes in Fluid Dynamics \\
\hline
115 & Finite Volume Method \\
\hline
116 & PHOENIX model analysis \\
\hline
117 & Beam Size and Efficiency Measurements \\
\hline
118 & CMB Analysis Techniques \\
\hline
119 & Effective Field Theory Methods \\
\hline
120 & EM Algorithm Applications \\
\hline
121 & CASA Data Reduction Techniques \\
\hline
122 & Strömgren magnitude correction \\
\hline
123 & Terahertz Observational Methods \\
\hline
124 & Granulation Analysis Techniques \\
\hline
125 & Astrophysical Observational Methods \\
\hline
126 & WKB approximation analysis \\
\hline
127 & Wiener Filter Methodology \\
\hline
128 & BAO Distance Measurements \\
\hline
129 & Infrared Imaging with HgCdTe Detectors \\
\hline
130 & Spectroscopic parameter fitting \\
\hline
131 & Regularization Techniques in Optimization \\
\hline
132 & Spectral Analysis and Modeling \\
\hline
133 & Error Budget Analysis \\
\hline
134 & Gravitational Wave Analysis \\
\hline
135 & LIGO data analysis methods \\
\hline
136 & Coriolis force analysis \\
\hline
137 & Calibration and Imaging with Difmap \\
\hline
138 & Vorticity Analysis and Simulation \\
\hline
139 & Seismic analysis methods \\
\hline
140 & Image Analysis and Data Processing \\
\hline
141 & EUV Wave Analysis \\
\hline
142 & Heterodyne Instrument Methodology \\
\hline
143 & Astrophysical Data Analysis \\
\hline
144 & Cluster Richness Estimation \\
\hline
145 & Advection Scheme Methodologies \\
\hline
146 & Autocovariance Analysis \\
\hline
147 & Sigma Clipping Procedures \\
\hline
148 & Lagrangian field theory \\
\hline
149 & Lagrangian Perturbation Theory \\
\hline
150 & Inflationary Model Analysis \\
\hline
151 & Spectral Line Analysis \\
\hline
152 & Infrared Space Observations \\
\hline
153 & Bulge Density Analysis \\
\hline
154 & Random Forest Classification \\
\hline
155 & HIRES Spectroscopy Techniques \\
\hline
156 & Age-dating methodologies \\
\hline
157 & Aerosol Characterization Methods \\
\hline
158 & Ionospheric Correction Techniques \\
\hline
159 & Precipitable Water Vapor Analysis \\
\hline
160 & Response File Generation \\
\hline
161 & Gauge Invariance Methodology \\
\hline
162 & Spectrogram Analysis Techniques \\
\hline
163 & IFU Spectroscopy Methodology \\
\hline
164 & Monte Carlo simulations \\
\hline
165 & Radioactive decay analysis methods \\
\hline
166 & Herschel-HIFI Spectral Analysis \\
\hline
167 & Riemann solver methodologies \\
\hline
168 & PPN Parameter Analysis \\
\hline
169 & CME observation and modeling \\
\hline
170 & Spectral Analysis Techniques \\
\hline
171 & PMS star analysis methods \\
\hline
172 & Orbital Dynamics Simulations \\
\hline
173 & Thermal Control Methodologies \\
\hline
174 & Geodesic Equation Analysis \\
\hline
175 & Observational and Theoretical Analysis \\
\hline
176 & Outburst Analysis Techniques \\
\hline
177 & Reflectance Spectra Analysis \\
\hline
178 & Time Lag Analysis \\
\hline
179 & Magnetization Parameter Simulations \\
\hline
180 & Collimator and Detector Methodologies \\
\hline
181 & Planetary formation simulations \\
\hline
182 & Optical Mirror Systems \\
\hline
183 & Tully-Fisher Relation Analysis \\
\hline
184 & Barycentric Corrections and Velocities \\
\hline
185 & Data Screening Criteria \\
\hline
186 & Numerical simulation of fluid dynamics \\
\hline
187 & Convection Modeling Techniques \\
\hline
188 & Gaussian Process Regression \\
\hline
189 & Kernel-based Smoothing Methods \\
\hline
190 & CCD Image Calibration Techniques \\
\hline
191 & Irradiance Measurement Techniques \\
\hline
192 & DIRBE and COBE Data Analysis \\
\hline
193 & Nucleosynthesis Calculations \\
\hline
194 & Reaction Rate Calculations \\
\hline
195 & HI emission analysis \\
\hline
196 & Wavefront Sensor Techniques \\
\hline
197 & Numerical simulations of tidal interactions \\
\hline
198 & Emission-line object identification \\
\hline
199 & Markov Chain Monte Carlo \\
\hline
200 & Observational Analysis Techniques \\
\hline
201 & Digital Filterbank Systems \\
\hline
202 & BL Lac object observations \\
\hline
203 & Proton Density and Energy Analysis \\
\hline
204 & Navier-Stokes Equations Analysis \\
\hline
205 & Power-Law Model Analysis \\
\hline
206 & Photochemical Modeling Techniques \\
\hline
207 & Spectral Analysis of FSRQs and BL Lacs \\
\hline
208 & Data Reduction and Analysis Techniques \\
\hline
209 & Cosmological MHD Simulations \\
\hline
210 & Diverse astrophysical methodologies \\
\hline
211 & Submillimeter Array Observations \\
\hline
212 & Dust Sublimation Modeling \\
\hline
213 & Blazar observation and analysis \\
\hline
214 & Geographical Hemisphere Analysis \\
\hline
215 & CMB Anisotropy Analysis \\
\hline
216 & Jet modeling and analysis \\
\hline
217 & Jeans Equation Analysis \\
\hline
218 & Power Spectrum Analysis \\
\hline
219 & Adaptive Optics Methodology \\
\hline
220 & VLBA Observational Techniques \\
\hline
221 & N-body simulations \\
\hline
222 & Spherical Harmonics Analysis \\
\hline
223 & Heat Flux Analysis \\
\hline
224 & Time Transformation Methods \\
\hline
225 & Likelihood Ratio Analysis \\
\hline
226 & Finite Difference Methods \\
\hline
227 & Spectral Line Profile Analysis \\
\hline
228 & Magnetospheric modeling techniques \\
\hline
229 & Isotopic Ratio Analysis \\
\hline
230 & Radio Frequency Interference Analysis \\
\hline
231 & Autocorrelator Data Acquisition \\
\hline
232 & Charge Transfer Inefficiency Correction \\
\hline
233 & Black Hole Mass Estimation \\
\hline
234 & N-body simulations \\
\hline
235 & MRI turbulence simulations \\
\hline
236 & Sunspot Number Calculation \\
\hline
237 & Fringe Fitting Techniques \\
\hline
238 & Comet Observation and Analysis \\
\hline
239 & Perturbation Theory Analysis \\
\hline
240 & Statistical Significance Testing \\
\hline
241 & Frequency Analysis and Prewhitening \\
\hline
242 & Extragalactic Database Utilization \\
\hline
243 & Photodissociation and Emission Analysis \\
\hline
244 & Poisson process modeling \\
\hline
245 & Calorimeter and Tracker Systems \\
\hline
246 & ROSAT PSPC X-ray Analysis \\
\hline
247 & Cubic and B-spline methods \\
\hline
248 & BeppoSAX Instrument Analysis \\
\hline
249 & Virgo Cluster Observational Studies \\
\hline
250 & Hyperfine Structure Analysis \\
\hline
251 & Magnetic Reconnection Studies \\
\hline
252 & Visibility Data Analysis \\
\hline
253 & Sound Speed Analysis \\
\hline
254 & Field Equation Manipulation \\
\hline
255 & Cross-Correlation Function Analysis \\
\hline
256 & Spectral Analysis of NH3 \\
\hline
257 & Molecular Line Observations \\
\hline
258 & Cumulative Distribution Function Analysis \\
\hline
259 & CME velocity and propagation analysis \\
\hline
260 & Relativistic Fluid Dynamics \\
\hline
261 & Broad Line Region Modeling \\
\hline
262 & Merger modeling and simulations \\
\hline
263 & Mass Transfer Modeling \\
\hline
264 & Beam Width Measurements \\
\hline
265 & HEAsoft Data Reduction \\
\hline
266 & Numerical Dynamo Simulations \\
\hline
267 & Adaptive Mesh Refinement Techniques \\
\hline
268 & Godunov Method Applications \\
\hline
269 & Spectral Analysis of PAH Emission \\
\hline
270 & X-ray spectral analysis \\
\hline
271 & Ray Tracing Simulations \\
\hline
272 & HII region analysis methods \\
\hline
273 & Clustering Analysis Methods \\
\hline
274 & Modified Gravity Theories \\
\hline
275 & Dark Matter Mass Analysis \\
\hline
276 & NLTE corrections and abundances \\
\hline
277 & RXTE Data Analysis \\
\hline
278 & Potential Function Analysis \\
\hline
279 & Data Reconstruction and Analysis \\
\hline
280 & Solar Motion Corrections \\
\hline
281 & Galactocentric distance analysis \\
\hline
282 & Gravitational Potential Analysis \\
\hline
283 & Alfvén Speed Analysis \\
\hline
284 & Hydrodynamical simulations \\
\hline
285 & Molecular Cloud Observations \\
\hline
286 & Kepler mission data analysis \\
\hline
287 & Ideal Gas Law Analysis \\
\hline
288 & Cooling Function Analysis \\
\hline
289 & GILDAS software data analysis \\
\hline
290 & Sérsic profile fitting \\
\hline
291 & Cosmic Microwave Background Analysis \\
\hline
292 & HST WFC3 Observational Analysis \\
\hline
293 & Viscosity Modeling in Discs \\
\hline
294 & Histogram Analysis Techniques \\
\hline
295 & Photospheric Analysis Methods \\
\hline
296 & Solar Activity Analysis \\
\hline
297 & Stellar Evolution and Burning Phases \\
\hline
298 & Cosmic Variance Analysis \\
\hline
299 & Microlensing Event Analysis \\
\hline
300 & Radio Astronomy Observations \\
\hline
301 & Neutron Capture and Fission Analysis \\
\hline
302 & CIAO data processing \\
\hline
303 & SMBH mass estimation methods \\
\hline
304 & Leptonic emission modeling \\
\hline
305 & Point Source Catalog Analysis \\
\hline
306 & Kolmogorov-Smirnov Test Applications \\
\hline
307 & Multi-Frame Blind Deconvolution \\
\hline
308 & Electron Temperature Estimation \\
\hline
309 & Metric Perturbation Analysis \\
\hline
310 & Photometric Distance Estimation \\
\hline
311 & Hadronic Interaction Modeling \\
\hline
312 & Emissivity and Correlation Analysis \\
\hline
313 & Acousto-Optical Spectrometry \\
\hline
314 & Chandra ACIS observations \\
\hline
315 & Covariance Matrix Analysis \\
\hline
316 & Pulse Profile Analysis \\
\hline
317 & Stellar Evolution and Observations \\
\hline
318 & Phase Calibration Observations \\
\hline
319 & Equation of State Analysis \\
\hline
320 & Atmospheric Parameter Analysis \\
\hline
321 & Linear Ephemeris Analysis \\
\hline
322 & Cherenkov light analysis \\
\hline
323 & Accretion Rate Analysis \\
\hline
324 & Spectral Normalization Techniques \\
\hline
325 & Wavelet Transform Analysis \\
\hline
326 & Flat-fielding and Calibration Techniques \\
\hline
327 & Gamma/Hadron Discrimination Techniques \\
\hline
328 & Photomultiplier Tube Detection \\
\hline
329 & Roche Lobe Analysis \\
\hline
330 & HST image processing \\
\hline
331 & VLT Spectroscopy Techniques \\
\hline
332 & Spacecraft Observation Methodology \\
\hline
333 & Halo modeling and simulations \\
\hline
334 & WFPC2 Imaging Analysis \\
\hline
335 & Herschel Space Observatory methods \\
\hline
336 & Magnetogram Data Processing \\
\hline
337 & Thermal Plasma Modeling \\
\hline
338 & Bootstrap Resampling Techniques \\
\hline
339 & Atmospheric Imaging Assembly (AIA) Analysis \\
\hline
340 & Mass-loss rate modeling \\
\hline
341 & 1D LTE and non-LTE analysis \\
\hline
342 & Time Series Analysis \\
\hline
343 & Spectropolarimetric Data Analysis \\
\hline
344 & H II Region Analysis \\
\hline
345 & Dust Opacity Calculations \\
\hline
346 & H i emission analysis \\
\hline
347 & Shower Core Analysis Techniques \\
\hline
348 & Bispectrum Estimation Methods \\
\hline
349 & Correlation Function Analysis \\
\hline
350 & Rotational Broadening Analysis \\
\hline
351 & Neutrino event analysis \\
\hline
352 & Statistical Estimation Techniques \\
\hline
353 & AGB Star Analysis \\
\hline
354 & QPO Frequency Analysis \\
\hline
355 & Polarimetric Data Analysis \\
\hline
356 & VLA Observational Methodology \\
\hline
357 & Self-calibration techniques \\
\hline
358 & Thermal Heating Simulations \\
\hline
359 & GRB analysis methodologies \\
\hline
360 & Halo mass analysis methods \\
\hline
361 & Spectral and Spatial Sampling Methods \\
\hline
362 & Pulsar Timing Analysis \\
\hline
363 & Chopper-wheel calibration method \\
\hline
364 & Dust Temperature Estimation \\
\hline
365 & Foreground Subtraction Techniques \\
\hline
366 & Gravity Darkening and Albedo Assumptions \\
\hline
367 & Probability Density Function Analysis \\
\hline
368 & Column Density Measurements \\
\hline
369 & AIPS Data Processing \\
\hline
370 & Astronomical Image Processing \\
\hline
371 & Confidence Interval Analysis \\
\hline
372 & Spectrograph data acquisition \\
\hline
373 & XSPEC spectral fitting methods \\
\hline
374 & Photoionization and Photoevaporation Analysis \\
\hline
375 & Fourier Transform Analysis \\
\hline
376 & Type Ia Supernova Analysis \\
\hline
377 & Brightness Temperature Analysis \\
\hline
378 & HST WFPC2 Data Processing \\
\hline
379 & Equivalent Width Measurements \\
\hline
380 & Color-Magnitude Diagram Analysis \\
\hline
381 & Accretion Rate Modeling \\
\hline
382 & Black Hole Mass and Spin Estimation \\
\hline
383 & Balmer line analysis \\
\hline
384 & Spectral Line Observations \\
\hline
385 & Comptonization modeling \\
\hline
386 & Synchrotron Emission Modeling \\
\hline
387 & Fermi Science Tools Analysis \\
\hline
388 & Correlator configuration methods \\
\hline
389 & Photometric Analysis of RR Lyrae \\
\hline
390 & XMM/EPIC Spectral Analysis \\
\hline
391 & Magnetic Field Analysis \\
\hline
392 & Neutron Star Mass Estimation \\
\hline
393 & Point Spread Function Analysis \\
\hline
394 & Pulsar Timing Analysis \\
\hline
395 & Photometric Analysis Techniques \\
\hline
396 & Background Subtraction Techniques \\
\hline
397 & Spectropolarimetric Observations \\
\hline
398 & Stokes Parameter Analysis \\
\hline
399 & Spectral Line Analysis \\
\hline
400 & ALMA observational configurations \\
\hline
401 & Evolutionary Track Analysis \\
\hline
402 & Monte Carlo simulations \\
\hline
403 & Virial Theorem Applications \\
\hline
404 & Spectral Line Fitting Techniques \\
\hline
405 & RXTE/PCA Galactic Emission Analysis \\
\hline
406 & Chandra spectral analysis \\
\hline
407 & Isophote Analysis Techniques \\
\hline
408 & CCD camera imaging techniques \\
\hline
409 & Radiation Field Modeling \\
\hline
410 & Magnetic Field Measurement Techniques \\
\hline
411 & Dark Energy Modeling \\
\hline
412 & Astrometric Calibration Techniques \\
\hline
413 & Observational Flux Measurements \\
\hline
414 & HST Camera Observations \\
\hline
415 & Mass Density Measurements \\
\hline
416 & Sky Background Subtraction Techniques \\
\hline
417 & Friedmann Equation Analysis \\
\hline
418 & Cosmological Parameter Estimation \\
\hline
419 & Light Curve Analysis \\
\hline
420 & Chandra data analysis using CIAO \\
\hline
421 & Lambda CDM model analysis \\
\hline
422 & Wavelength Calibration Methods \\
\hline
423 & Observational Data Analysis \\
\hline
424 & Supernova Light Curve Analysis \\
\hline
425 & Spectral Analysis Techniques \\
\hline
426 & RPC Detector Methodology \\
\hline
427 & Limb Darkening Coefficients Analysis \\
\hline
428 & Null Hypothesis Testing \\
\hline
429 & Ideal MHD simulations \\
\hline
430 & EPIC camera data analysis \\
\hline
431 & Scattered Light Analysis \\
\hline
432 & Hubble parameter analysis \\
\hline
433 & Common Astronomy Software Applications \\
\hline
434 & Reddening Correction Methods \\
\hline
435 & XMM-Newton Data Analysis \\
\hline
436 & XMM-Newton EPIC observations \\
\hline
437 & Spectrophotometric Calibration Methods \\
\hline
438 & Magnetohydrodynamics Modeling \\
\hline
439 & Spectral Fitting with XSPEC \\
\hline
440 & Power Law Spectral Fitting \\
\hline
441 & Instrument Specifications and FOV \\
\hline
442 & Star Formation Rate Modeling \\
\hline
443 & Cosmological Parameter Estimation \\
\hline
444 & Antenna Temperature Calibration \\
\hline
445 & Spectroscopic Observational Methods \\
\hline
446 & Cosmic Ray Data Processing \\
\hline
447 & Main Beam Brightness Temperature \\
\hline
448 & Spectral Calibration Techniques \\
\hline
449 & Abundance Determination Methods \\
\hline
450 & Velocity Measurement Techniques \\
\hline
451 & Sky Subtraction Techniques \\
\hline
452 & Simulation particle mass resolutions \\
\hline
453 & IRAC and MIPS Data Analysis \\
\hline
454 & Redshift Extraction and Analysis \\
\hline
455 & Radial Velocity Measurements \\
\hline
456 & Statistical Analysis of GRB Data \\
\hline
457 & Colour and Magnitude Analysis \\
\hline
458 & Spectroscopic Grism Observations \\
\hline
459 & Flux Measurement and Analysis \\
\hline
460 & Chandra X-ray Observations \\
\hline
461 & XMM-Newton Data Reduction \\
\hline
462 & Sloan Digital Sky Survey methods \\
\hline
463 & Padova Isochrone Fitting \\
\hline
464 & CDM model analysis \\
\hline
465 & Dark Energy Parameterization \\
\hline
466 & Coordinate System Definitions \\
\hline
467 & Spectrum Extraction Methods \\
\hline
468 & Supernova Luminosity Distance Analysis \\
\hline
469 & Spectral Line Observations \\
\hline
470 & 2MASS Data Utilization \\
\hline
471 & 2MASS data analysis \\
\hline
472 & Surface Brightness Measurements \\
\hline
473 & Parameter Variation Analysis \\
\hline
474 & Metallicity and Abundance Analysis \\
\hline
475 & Cosmological Parameter Estimation \\
\hline
476 & Imaging and Noise Analysis \\
\hline
477 & Transit Modeling Parameters \\
\hline
478 & X-ray and gamma-ray observations \\
\hline
479 & MESA stellar evolution modeling \\
\hline
480 & Orbital Parameter Analysis \\
\hline
481 & Light Curve Analysis \\
\hline
482 & Photometric Data Analysis \\
\hline
483 & Lightcurve Analysis and Observation \\
\hline
484 & Grid-based simulation methodologies \\
\hline
485 & Amplitude Spectrum Analysis \\
\hline
486 & Magnitude-based galaxy selection \\
\hline
487 & Spectroscopic and Photometric Redshifts \\
\hline
488 & Magnitude Measurement and Analysis \\
\hline
489 & Bayesian inference with MCMC \\
\hline
490 & Count Rate Analysis \\
\hline
491 & Spectra Extraction Methodology \\
\hline
492 & Ly Alpha Emission Analysis \\
\hline
493 & Spectral Analysis of Emission/Absorption \\
\hline
494 & Chi-squared statistical analysis \\
\hline
495 & Radiative Transfer Methods \\
\hline
496 & Bayesian Inference Methods \\
\hline
497 & Likelihood-based parameter estimation \\
\hline
498 & Statistical Analysis and Fitting \\
\hline
499 & Light Curve Analysis \\
\hline
500 & Isochrone Age Estimation \\
\hline
501 & Cosmological simulations and modeling \\
\hline
502 & Stellar PSF Processing Techniques \\
\hline
503 & Cosmological Parameter Estimation \\
\hline
504 & Color and Magnitude Calibration \\
\hline
505 & Power Spectrum Analysis \\
\hline
506 & Gravitational Lensing Analysis \\
\hline
507 & CCD Imaging Techniques \\
\hline
508 & Spectrophotometric Flux Calibration \\
\hline
509 & Photometric Analysis Techniques \\
\hline
510 & Circular Region Analysis \\
\hline
511 & Astrometric Measurements \\
\hline
512 & PSF Fitting and Correction Methods \\
\hline
513 & CMB Observations and Analysis \\
\hline
514 & Point Spread Function Analysis \\
\hline
515 & Reionization Modeling Techniques \\
\hline
516 & Photoionization Modeling Techniques \\
\hline
517 & Power Spectrum Analysis \\
\hline
518 & Telescope Observations and Simulations \\
\hline
519 & Spectral Analysis and Fitting \\
\hline
520 & PSF-fitting photometry methods \\
\hline
521 & Stellar Isochrone Analysis \\
\hline
522 & Photometric Redshift Analysis \\
\hline
523 & Luminosity and Mass Estimation \\
\hline
524 & PSF Fitting in Photometry \\
\hline
525 & Stellar SED Analysis \\
\hline
526 & Luminosity and Scaling Relations \\
\hline
527 & Luminosity Distance Calculations \\
\hline
528 & Aperture Photometry Techniques \\
\hline
529 & Photometric Analysis Techniques \\
\hline
530 & IRAF Data Reduction \\
\hline
531 & IRAF Aperture Photometry \\
\hline
\end{tabular}
\end{table}